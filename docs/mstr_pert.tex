% A write up of the derivation of a perturbative master equation for
% the Transverse-field ising model
\documentclass[a4paper]{memoir}
\usepackage{amsmath,amsfonts,amssymb}
\usepackage{minted}
\newcommand{\bra}[1]{\langle#1|}
\newcommand{\ket}[1]{|#1\rangle}
\newcommand{\gap}{\,\,\,\,\,\,\,\,\,\,\,}
\newcommand{\Choose}[2]{{}^{#1}\hspace{-1pt}C_{#2}}


\newcommand{\enn}{\epsilon_{n,i_n}}
\newcommand{\enm}{\epsilon_{m,i_m}}
\newcommand{\ennpert}[1]{\epsilon^{(#1)}_{n,i_n}}


\newcommand{\enrnought}{\epsilon_{r,i_r}^{(0)}}
\newcommand{\ennnought}{\epsilon_{n,i_n}^{(0)}}
\newcommand{\enmnought}{\epsilon_{m,i_m}^{(0)}}
\newcommand{\enmpnought}{\epsilon_{m+1,i_{m+1}}^{(0)}}
\newcommand{\enmpppnought}{\epsilon_{m+3,i_{m+3}}^{(0)}}
\newcommand{\enmmnought}{\epsilon_{m-1,i_{m-1}}^{(0)}}

\newcommand{\anr}{a^{n,i_n}_{r,i_r}}
\newcommand{\anrnought}{\delta^n_r\delta^{i_n}_{i_r}}
\newcommand{\anrpert}[1]{a^{n,i_n,(#1)}_{r,i_r}}

\newcommand{\sigpp}{\hat{\Sigma}^{(+2)}}
\newcommand{\sigmm}{\hat{\Sigma}^{(-2)}}

\newcommand{\bigSig}{\tilde{\mathbf{\Sigma}}^{(l)}_m}
\newcommand{\bigSigp}{\tilde{\mathbf{\Sigma}}^{(l)}_{m+1}}
\newcommand{\bigSigpp}{\tilde{\mathbf{\Sigma}}^{(l)}_{m+2}}
\newcommand{\bigSigmm}{\tilde{\mathbf{\Sigma}}^{(l)}_{m-2}}

%---------------------------------------------------------------------

\setlrmarginsandblock{1.6in}{1.6in}{*}
\checkandfixthelayout

\begin{document}
%
\subsection*{Introduction}
%
This document aims to explain the numerical solution of the
Born-Markov master equation, for a transverse-field $XY$ model, close
to the $XY$-$XX$ crossover. First is an explanation of the model, the
eigenstates of the unperturbed case, which have a spin-subspace
structure which allows for quick exact diagonalisation. Then the
perturbative corrections are derived, and the numerical algorithm is
described.
%
\subsection*{Hamiltonian}
%
The system of interest is a spin-$\frac{1}{2}$ chain of length $N$
coupled to a set of $N$ independent thermal environments, one for each
spin. The system is modelled using an (anti-)ferromagnetic
Hamiltonian, describing nearest-neighbour interactions with hopping
strength $-J$. There is also an external magnetic field, whose
magnitude is characterised by a constant $g$. Finally, there is
pair-creation and annihilation induced by a small pertubative factor
$\lambda \ll J,g$.
%
\par It is of interest to calculate the `eigenoperators' $A(\omega)$
of this system. This allows for the modelling of the evolution of the
system under the infulence of the thermal environments, with a
Born-Markov master equation. This however, requires knowing the
eigenstates of the system, $\ket{\epsilon}$. Now the Hamiltonian of
the system will be described.
%
\par The spins are modelled using Pauli operators $\sigma^\mu$, $\mu =
x,y,z,+,-$ and so the system is described by a Hamiltonian
\begin{align}
  H&=H_0+\lambda \hat{V}\notag \\
   &=-J\sum_n(\sigma^+_n\sigma^-_{n+1}+\sigma^-_n\sigma^+_{n+1})+gS^z+
  \lambda\sum_n(\sigma^+_n\sigma^+_{n+1}+\sigma^-_n\sigma^-_{n+1}),
\end{align}
where $S^z = \sum_n \sigma_n^z$. The sum over $n$ takes all values of
the site label with the inclusion of periodic boundary conditions.
Since the total $z$-projection spin operator $S^z$ commutes with the
unperturbed Hamiltonian $[H_0,S^z]=0$, we may order the eigenstates in
terms of their spin-sector. Let $\ket{\enrnought}$ denote the
${i_r}^\text{th}$ eigenstate with $r$ up-spins of the unperturbed
Hamiltonian. Then we have:
\begin{align}
  H_0\ket{\enrnought}=\epsilon_{r,i_r}^{(0)}\ket{\enrnought},\gap
  S^z\ket{\enrnought}=(2r-N)\ket{\enrnought}.
\end{align}
% 
The label $r$ runs from $0$ to $N$, giving a total of $N+1$
spin-sectors. In each subsector, there are $r$ up-spins distributed on
a chain of $N$ spins, which gives an $\Choose{N}{r}$-dimensional
space, where $\Choose{N}{r}$ is the binomial coefficient
$N!/(N-r)!r!$. Therefore, $i_r$ runs from $1$ to $\Choose{N}{r}$.
%
\par Consider the configuration basis, ordered in terms of the total
$z$-spin projection, which for $N=3$ is
\begin{align}
  \ket{000},\ket{100},\ket{010},\ket{001},\ket{110},\ket{101},\ket{011},\ket{111}.
\end{align}
Let $\ket{\{\sigma\}}$ denote one such configuration-basis state. Then
in this basis, $H_0$ takes a block-diagonal form:
%% +-----+-----+-----+-----+-----+-----+-----+-----+
%% |-3g  |0    |0    |0    |0    |0    |0    |0    |
%% +-----+-----+-----+-----+-----+-----+-----+-----+
%% |0    |-g   |-J   |-J   |0    |0    |0    |0    |
%% +-----+-----+-----+-----+-----+-----+-----+-----+
%% |0    |-J   |-g   |-J   |0    |0    |0    |0    |
%% +-----+-----+-----+-----+-----+-----+-----+-----+
%% |0    |-J   |-J   |-g   |0    |0    |0    |0    |
%% +-----+-----+-----+-----+-----+-----+-----+-----+
%% |0    |0    |0    |0    |g    |-J   |-J   |0    |
%% +-----+-----+-----+-----+-----+-----+-----+-----+
%% |0    |0    |0    |0    |-J   |g    |-J   |0    |
%% +-----+-----+-----+-----+-----+-----+-----+-----+
%% |0    |0    |0    |0    |-J   |-J   |g    |0    |
%% +-----+-----+-----+-----+-----+-----+-----+-----+
%% |0    |0    |0    |0    |0    |0    |0    |3g   |
%% +-----+-----+-----+-----+-----+-----+-----+-----+
\begin{align}
  \bra{\{\sigma\}}H_0\ket{\{\sigma'\}}=\left(
  \begin{array}{cccccccc}
    -3g & 0 & 0 & 0 & 0 & 0 & 0 & 0 \\
    0 & -g & -J & -J & 0 & 0 & 0 & 0 \\
    0 & -J & -g & -J & 0 & 0 & 0 & 0 \\
    0 & -J & -J & -g & 0 & 0 & 0 & 0 \\
    0 & 0 & 0 & 0 & g & -J & -J & 0 \\
    0 & 0 & 0 & 0 & -J & g & -J & 0 \\
    0 & 0 & 0 & 0 & -J & -J & g & 0 \\
    0 & 0 & 0 & 0 & 0 & 0 & 0 & 3g \\
  \end{array}
  \right).
\end{align}
The eigenvectors of this matrix can be found by diagonalising the four
sub-blocks of the matrix.
%
\subsection*{Perturbation theory}
%
Next, the perturbative corrections to the eigenstates of the
unperturbed Hamiltonian are calculated to give an approximation to the
full perturbed model.
\par To begin, we note that the unperturbed eigenstates form a
complete state and so any state in the system space may be expressed
as a superposition
\begin{align}
  \ket{\psi}=\sum_{r=0}^N\sum_{i_r=1}^{\Choose{N}{r}}a_{r,i_r}\ket{\enrnought},
\end{align}
where the $a_{r,i_r}$ are complex numbers. This principle of
superposition must also apply to the eigenstates of the perturbed
Hamiltonian too:
\begin{align}
  \ket{\enn}=\sum_{r}\sum_{i_r}\anr\ket{\enrnought}.
\end{align}
In this expression the symbol $\anr$ is represented as a perturbation
series
\begin{align}
  \anr = \anrnought+\lambda\anrpert{1} + \lambda^2\anrpert{2} +
  \mathcal{O}(\lambda^3).
\end{align}
Similarly the perturbed eigenenergies can be written
\begin{align}
  \enn = \ennnought+\lambda\ennpert{1}+\lambda^2\ennpert{2}
         +\mathcal{O}(\lambda^3).
\end{align}
%
\par Perturbation theory allows the calculation of the terms $\anrpert{m}$
and $\ennpert{m}$ in terms of the unperturbed eigenenergies and
unperturbed eigenstates. Without going into the full derivation (which
is simply a term-by-term equation of the Schr\"odinger equation using
the perturbation series), the perturbative corrections are found to be,
at first order:
\begin{align}
  \ennpert{1}=\bra{\ennnought}\hat{V}\ket{\ennnought},\gap
  \anrpert{1}=\frac{\bra{\enrnought}\hat{V}\ket{\ennnought}}
          {\ennnought-\enrnought}.
\end{align}
The remainder of this section will involve rewriting these expressions
-- particularly the first order correction to the wavefunctions -- in
terms of the subspace structure, so that a matrix form may be written
down.
%
\subsubsection*{Rewriting perturbative correction}
%
Let us first devide up the perturbative term in the Hamiltonian into
two operators:
\begin{align}
  \hat{V} = \sum_n(\sigma^+_n\sigma^+_{n+1}+\sigma^-_n\sigma^-_{n+1})
          = \sigpp+\sigmm,
\end{align}
where we have defined $\sigpp\equiv\sum_n\sigma^+_n\sigma^+_{n+1}$ and
$\sigmm\equiv\sum_n\sigma^-_n\sigma^-_{n+1}$. Clearly $\sigpp$ will
take a state lying in the $r$-spin subspace to the $(r+2)$-spin
subspace and $\sigmm$ will take an $r$-spin state to an $(r-2)$-spin
state. More precisely:
\begin{align}
  \hat{V}\ket{\enrnought}=\sum_{i_{r-2}}^{\Choose{N}{r-2}}c_{i_{r-2}}
  \ket{\epsilon_{r-2,i_{r-2}}^{(0)}}
  +\sum_{i_{r+2}}^{\Choose{N}{r+2}}c_{i_{r+2}}
  \ket{\epsilon_{r+2,i_{r+2}}^{(0)}}.
\end{align}
Now, since the unperturbed eigenstates are an orthonormal set, we have
that
\begin{align}
  \langle\epsilon_{n,i_n}^{(0)}\ket{\enrnought}=\delta^n_r\delta_{i_r}^{i_n},
\end{align}
which implies immediately that
$\bra{\ennnought}\hat{V}\ket{\enrnought}=0$ so there is \textit{no}
energy correction at first order for this perturbation.
%
\par Since the perturbation splits into to two terms, we shall deal
with them independently. We first define the following symbols:
\begin{align}
  K_{r,n}^{(i_r,i_n)}=\frac{\bra{\enrnought}\sigmm\ket{\ennnought}}
                           {\ennnought-\enrnought},\gap
  M_{r,n}^{(i_r,i_n)}=\frac{\bra{\enrnought}\sigpp\ket{\ennnought}}
                           {\ennnought-\enrnought}.
\end{align}
With these new definitions the perturbative expansion of the
eigenstates becomes:
\begin{align}
  \ket{\enn} = \ket{\ennnought}+\lambda\sum_r\sum_{i_r}
  \left(K_{r,n}^{(i_r,i_n)}+M_{r,n}^{(i_r,i_n)}\right)\ket{\enrnought}
  +\mathcal{O}(\lambda^2).
\end{align}
We may simplify this by noting that because $\sigmm$ and $\sigpp$
remove or add 2 spins, the tensor
$K_{r,n}^{(i_r,i_n)}+M_{r,n}^{(i_r,i_n)}$ has a value of zero if
$|r-n|>2$. This is encapsulated by the following expressions:
\begin{align}
  K_{r,n}^{(i_r,i_n)}=\delta_{r,n-2}K_{n-2,n}^{(i_{n-2},i_n)}\gap
  M_{r,n}^{(i_r,i_n)}=\delta_{r,n+2}M_{n+2,n}^{(i_{n+2},i_n)}
\end{align}
which leads to the introduction of the symbols
\begin{align}
  K_{n-2}^{(i,j)}=K_{n-2,n}^{(i_{n-2},j_n)}\gap\text{and}\gap
  M_{n+2}^{(i,j)}=M_{n+2,n}^{(i_{n+2},j_n)}.
\end{align}
The perturbative expansion then takes the form:
\begin{align}
  \ket{\enn} = \ket{\ennnought}+\lambda\left(
  \sum_{i}K_{n-2}^{(i,i_n)}\ket{\epsilon_{n-2,i}^{(0)}}
  +\sum_{i'}M_{n+2}^{(i',i_n)}\ket{\epsilon_{n+2,i'}^{(0)}}\right)
  +\mathcal{O}(\lambda^2).
\end{align}
%
\subsection{Evaluating matrix elements of $\sigma_l^-$}
%
It is required that the matrix element
$\bra{\enm}\sigma_l^-\ket{\enn}$ be calculated so that the terms in
the master equation can be evaluated numerically. The first step is to
use the pertubative expansion of the eigenstates to express the matrix
element to first order. To this end, we shall ignore the cross terms
which give order $\lambda^2$ or higher.
\begin{align}
  \bra{\enm}\sigma_l^-\ket{\enn} &= \left[\bra{\enmnought} + \lambda
    \left( \sum_{i}{K_{m-2}^{(i,i_m)}}^* \bra{\epsilon_{m-2,i}^{(0)}}
    +\sum_{i'}{M_{m+2}^{(i',i_m)}}^* \bra{\epsilon_{m+2,i'}^{(0)}}
    \right)\right]\notag \\ &\times\sigma_l^-\left[\ket{\ennnought} +
    \lambda \left( \sum_{i}K_{n-2}^{(i,i_n)}
    \ket{\epsilon_{n-2,i}^{(0)}} +\sum_{i'}M_{n+2}^{(i',i_n)}
    \ket{\epsilon_{n+2,i'}^{(0)}} \right)\right] \notag \\
\end{align}
This equation is first expanded, then the order $\lambda^2$ terms are thrown away. What results is the expression:
\begin{align}
  &\bra{\enmnought}\sigma_l^-\ket{\ennnought}+
  \lambda\bra{\enmnought}\sigma_l^-\left( \sum_{i}K_{n-2}^{(i,i_n)}
  \ket{\epsilon_{n-2,i}^{(0)}} +\sum_{i'}M_{n+2}^{(i',i_n)}
  \ket{\epsilon_{n+2,i'}^{(0)}} \right) \notag \\ &\gap + \lambda
  \left( \sum_{i}{K_{m-2}^{(i,i_m)}}^* \bra{\epsilon_{m-2,i}^{(0)}}
  +\sum_{i'}{M_{m+2}^{(i',i_m)}}^* \bra{\epsilon_{m+2,i'}^{(0)}}
  \right)\sigma_l^-\ket{\ennnought},\notag
\end{align}
which after a simple rearrangement of the bras and kets yields
\begin{align}
\bra{\enm}\sigma_l^-\ket{\enn}&=\bra{\enmnought}\sigma_l^-\ket{\ennnought}+\lambda\bigg(\sum_{i}K_{n-2}^{(i,i_n)}
  \bra{\enmnought}\sigma_l^-\ket{\epsilon_{n-2,i}^{(0)}}
  +\sum_{i'}M_{n+2}^{(i',i_n)}
  \bra{\enmnought}\sigma_l^-\ket{\epsilon_{n+2,i'}^{(0)}}\notag\\ &\gap\,\,\,
  + \sum_{i}{K_{m-2}^{(i,i_m)}}^*
  \bra{\epsilon_{m-2,i}^{(0)}}\sigma_l^-\ket{\ennnought}
  +\sum_{i'}{M_{m+2}^{(i',i_m)}}^*
  \bra{\epsilon_{m+2,i'}^{(0)}}\sigma_l^-\ket{\ennnought}\bigg). \notag \\
\end{align}
This expression can be split into cases, since $\sigma_l^-$ will only
talk between eigenstates living in subspaces seperated by total spin
value of $1$. More precisely this is expressed as
$\bra{\enmnought}\sigma_l^-\ket{\ennnought}=\delta_{n,m+1}\bra{\enmnought}\sigma_l^-\ket{\enmpnought}$. Instead
of writing multiple Kronecker deltas on one line let us write the
result as a piecewise expression:
\begin{align}
  \bra{\enm}\sigma_l^-\ket{\enn}=
  \begin{cases}
    \text{if} \,n=m+1 & \bra{\enmnought}\sigma_l^-\ket{\enmpnought}, \\
    \text{if} \,n=m-1 & \sum_{i'}M_{m+1}^{(i',i_{m-1})}
              \bra{\enmnought}\sigma_l^-\ket{\epsilon_{m+1,i'}^{(0)}}
                      + \sum_{i}{K_{m-2}^{(i,i_m)}}^*
              \bra{\epsilon_{m-2,i}^{(0)}}\sigma_l^-\ket{\enmmnought}, \\
    \text{if} \,n=m+3 & \sum_{i}K_{m+1}^{(i,i_{m+3})}
              \bra{\enmnought}\sigma_l^-\ket{\epsilon_{m+1,i}^{(0)}}
                      + \sum_{i'}{M_{m+2}^{(i',i_m)}}^*
              \bra{\epsilon_{m+2,i'}^{(0)}}\sigma_l^-\ket{\enmpppnought}, \\
    \text{otherwise} & 0.
  \end{cases}
\end{align}
In an attempt to lighten the notation a bit, let the following
matrices be defined:
\begin{align}
  \bigSig=\big[\bra{\enmnought}\sigma_l^-\ket{\enmpnought}\big],\gap
  \mathbf{K}_r=[K_r^{(i,j)}],\gap\mathbf{M}_r=[M_r^{(i,j)}].
\end{align}
Which allows us to write the entries of the piecewise form above in a
matrix equation, since the sums over $i$ and $i'$ amount to performing
a matrix multiplication. Then for each pair, $(m,n)$ we construct the
following matrices:
\begin{align}
  \begin{tabular}{lll}
    \text{when} & $n=r+1$ & $\bigSig$, \\
    & $n=m-1$ & $\lambda\left[\bigSig\mathbf{M}_{m+1}^\top+\mathbf{K}_{r-2}^*\bigSigmm\right]$, \\
    & $n=m+3$ & $\lambda\left[\bigSig\mathbf{K}_{m+1}^\top+\mathbf{M}_{r-2}^*\bigSigmm\right]$, \\
    & otherwise & nothing.
  \end{tabular}
\end{align}
\end{document}
